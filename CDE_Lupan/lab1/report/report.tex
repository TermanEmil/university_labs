\documentclass{article}

\usepackage{enumitem}
\usepackage{listings}
\usepackage{color}
\usepackage{amsmath}
\usepackage{hyperref}
\usepackage{graphicx}
\usepackage{pgffor}
\usepackage{xparse}
\usepackage{expl3}
\usepackage{tabularx, makecell}
\usepackage{multirow}
\usepackage{booktabs}
\usepackage{indentfirst}
\usepackage{lipsum}
\usepackage{sectsty}
\usepackage[utf8]{inputenc}
\usepackage{csquotes}
\usepackage{xcolor}
\usepackage{fancyvrb}
\usepackage{fancyhdr}
\usepackage{fancyvrb}
\usepackage[most]{tcolorbox}
\usepackage{blindtext}
\usepackage{caption}
\usepackage{etoolbox}
\usepackage{booktabs}
\usepackage{karnaugh-map}
\usepackage{tikz}
\usepackage{mdframed}
\usepackage{calc}
\usepackage[nomessages]{fp}
\usepackage{pgfplots}
\usepackage{geometry}
\usepackage{python}
\usepackage{float}
\usepackage{colortbl}

\graphicspath{{./}}

\definecolor{codegreen}{rgb}{0,0.6,0}
\definecolor{codegray}{rgb}{0.5,0.5,0.5}
\definecolor{codepurple}{rgb}{0.58,0,0.82}
\definecolor{backcolour}{rgb}{0.95,0.95,0.92}

\sectionfont{\bfseries\Large\center} 

\geometry{
	a4paper,
	total={170mm,257mm},
	left=20mm,
	top=20mm,
}

\pgfplotsset{compat=1.13}

\begin{document}
	% Vars
	\newcommand{\ROne}{99.8}
	\newcommand{\RTwo}{198}
	\newcommand{\RThree}{53.7}
	\newcommand{\RMax}{473}
	
	\newcommand{\IPosOne}{42.1}
	\newcommand{\IPosTwo}{96.0}

	\newcommand{\UExOneOne}{4.24}
	\newcommand{\UExOneTwo}{8.43}
	\newcommand{\UExOneThree}{2.26}

	\newcommand{\IExTwoOne}{61.7}
	\newcommand{\IExTwoTwo}{43.2}
	\newcommand{\IExTwoThree}{18.6}

	\newcommand{\UExTwoOne}{6.66}
	\newcommand{\UExTwoTwo}{8.69}
	\newcommand{\UExTwoThree}{8.69}

	% <Square cases>
	\makeatletter
	\newenvironment{sqcases} {
		\matrix@check\sqcases\env@sqcases
	}{
		\endarray \right.
	}
	\def\env@sqcases {
		\let \@ifnextchar \new@ifnextchar
		\left \lbrack
		\def \arraystretch{1.2}
		\array{@{}l@{\quad}l@{}}
	}
	\makeatother
	% </Square cases>

	% My include img. (With frame and caption) 
	\newcommand{\myIncludeImg}[4]{
		\begin{center} \begin{figure}[H]
			\begin{mdframed}[backgroundcolor=black!5, rightline=false, leftline=false]
				\begin{center}
					\includegraphics[scale=#1]{#2}
					\caption{#3}
				\end{center}
			\end{mdframed}
			\label{fig:#4}
		\end{figure} \end{center}
	}

	% Begin of the document

	\title{CDE laboratory\_01}
	\author{Terman Emil FAF161}
	\maketitle

	% Write at bottom
	\vspace*{\fill}
	
	\centering
	\includegraphics{imgs/UTM_logo.png}

	\begin{flushright}
		Prof: O. Lupan
	\end{flushright}

	\LaTeX
	\pagebreak %-----------------------------------------------End of title page

	\raggedright
	\section{Verificarea îndeplinirii legilor lui Ohm si Kirchhoff pentru circuitele electrice neramificate si ramificate.}
		\myIncludeImg{0.7}{./imgs/ElectricCircuit1.jpg}{Circuit 1.1}{circuit1}

		\subsection{}
			\[
				\begin{tabular}{|c|c|c|c|c|c|c|c|}
					\hline
					$R_1 (\Omega)$ & $R_2 (\Omega)$ & $R_3 (\Omega)$ & $I_1 (mA)$ & $I_2 (mA)$ & $U_{t1} (V)$ & $U_{t2} (V)$\\
					\hline
					\ROne & \RTwo & \RThree & \IPosOne & \IPosTwo& 15.01 & 15.01\\
					\hline
				\end{tabular}
			\]

		\subsection{}
			\[
				r_0 = \frac{U_{t2} - U_{t1}}{I_1 - I_2} = \frac{0}{I_1 - I_2} = 0
			\]
			Aparatele nu au fost destul de fixe si $r_0$ este o valoare prea mica pentru a fi masurata exact.

		\subsection{}
			% Calculates the total sum of resistances
			\FPeval{\resistanceSum}{clip(\ROne + \RTwo + \RThree)}

			% 15 / (sum of resistance)
			\FPeval{\currentExThree}{clip(15 / \resistanceSum * 1000)}

			% Round \currentExThree to 2 fractional digits
			\FPround{\currentExThree}{\currentExThree}{2}

			\[
				I = \frac{E}{R_1 + R_2 + R_3 + r_0} = \frac{15}{\resistanceSum} = \currentExThree \ mA
			\]
			
			\FPeval{\UOneCalc}{clip(\currentExThree * \ROne / 1000)}
			\FPround{\UOneCalc}{\UOneCalc}{2}

			\FPeval{\UTwoCalc}{clip(\currentExThree * \RTwo / 1000)}
			\FPround{\UTwoCalc}{\UTwoCalc}{2}

			\FPeval{\UThreeCalc}{clip(\currentExThree * \RThree / 1000)}
			\FPround{\UThreeCalc}{\UThreeCalc}{2}

			\[U_1 = IR_1 = \currentExThree \cdot \ROne = \UOneCalc \ V\]
			\[U_2 = IR_2 = \currentExThree \cdot \RTwo = \UTwoCalc \ V\]
			\[U_3 = IR_3 = \currentExThree \cdot \RThree = \UThreeCalc \ V\]

		\subsection{}
			\begin{center} \begin{tabular}{|c|c|c|c|c|c|c|c|c|c|}
				\hline
				\multicolumn{2}{|c|}{R ($\Omega$)} &$
					I_{c}$(mA) &
					\multicolumn{2}{c|}{$U_c$ (V)} &
					$I_{m}$ (mA) &
					\multicolumn{2}{c|}{$U_m$ (V)} \\

				\hline
				
				$R_1$ & \ROne &
					\multirow{3}{*}{\currentExThree} &
					$U_1$ & \UOneCalc &
					\multirow{3}{*}{\IPosOne} &
					$U_1$ & \UExOneOne\\
				
				\cline{1-2} \cline{4-5} \cline{7-8}

				$R_2$ & \RTwo & & $U_2$ & \UTwoCalc & & $U_2$ & \UExOneTwo\\		
				\cline{1-2} \cline{4-5} \cline{7-8}
				
				$R_3$ & \RThree & & $U_3$ & \UThreeCalc & & $U_3$ & \UExOneThree\\
				\hline
			\end{tabular} \end{center}

		\subsection{}
			\FPeval{\UTotalCalc}{clip(\UOneCalc + \UTwoCalc + \UThreeCalc)}
			\FPeval{\UTotalMes}{clip(\UExOneOne + \UExOneTwo + \UExOneThree)}

			\[U_{c1} + U_{c2} + U_{c3} = \UOneCalc + \UTwoCalc + \UThreeCalc = \textbf{\UTotalCalc \ V}\]
			\[U_{m1} + U_{m2} + U_{m3} = \UExOneOne + \UExOneTwo + \UExOneThree = \textbf{\UTotalMes \ V}\]

		\subsection{}
			\myIncludeImg{0.3}{./imgs/PotentialChart.jpg}{Potential Chart of Circuit 1}{potentialChart}

		\subsection{}
			\myIncludeImg{0.7}{./imgs/ElectricCircuit2.jpg}{Circuit 1.2}{circuit2}
			
			\FPeval{\REtotal}{clip(\ROne + \RTwo * \RMax / (\RTwo + \RMax))}
			\FPround{\REtotal}{\REtotal}{2}
			\[
				R_E = R_1 + \frac{R_2 R_3}{R_2 + R_3} = \ROne + \frac{\RTwo \cdot \RMax}{\RTwo + \RMax} = \REtotal \ \Omega
			\]

			\FPeval{\IExTwoCalcOne}{clip(15 / \REtotal * 1000)}
			\FPround{\IExTwoCalcOne}{\IExTwoCalcOne}{2}

			\[
				I_1 = \frac{E}{r_0 + R_E} = \frac{15}{0 + \REtotal} = \IExTwoCalcOne \ mA
			\]

			\FPeval{\UExTwoCalcTwo}{clip(\IExTwoCalcOne * \RTwo * \RMax / (\RTwo + \RMax) / 1000)}
			\FPround{\UExTwoCalcTwo}{\UExTwoCalcTwo}{2}

			\[
				U_2 = U_3 = I_1 \frac{R_2 R_3}{R_2 + R_3} =
					\IExTwoCalcOne \cdot \frac{\RTwo \cdot \RMax}{\RTwo + \RMax} =
					\UExTwoCalcTwo \ V
			\]

			\FPeval{\UExTwoCalcOne}{clip(\IExTwoOne * \ROne / 1000)}
			\FPround{\UExTwoCalcOne}{\UExTwoCalcOne}{2}

			\[
				U_1 = I_1 \cdot R_1 = \IExTwoOne \cdot \ROne = \UExTwoCalcOne \ V
			\]

			%I_1
			\FPeval{\IExTwoCalcTwo}{clip(\UExTwoCalcTwo / \RTwo * 1000)}
			\FPround{\IExTwoCalcTwo}{\IExTwoCalcTwo}{2}

			%I_2
			\FPeval{\IExTwoCalcThree}{clip(\UExTwoCalcTwo / \RMax * 1000)}
			\FPround{\IExTwoCalcThree}{\IExTwoCalcThree}{2}

			\begin{center} \begin{tabular}{c|c}
				\(I_2 = \frac{U_2}{R_2} = \frac{\UExTwoCalcTwo}{\RTwo} = \IExTwoCalcTwo \ mA\)
				&
				\(I_3 = \frac{U_3}{R_3} = \frac{\UExTwoCalcTwo}{\RMax} = \IExTwoCalcThree \ mA\)\\
			\end{tabular} \end{center}

			\begin{center} \begin{tabular}{|c|c|c|c|c|c|c|c|c|c|}
				\hline
					\multicolumn{2}{|c|}{R ($\Omega$)} &
					\multicolumn{2}{c|}{$I_c (mA)$} &
					\multicolumn{2}{c|}{$U_c (V)$} &
					\multicolumn{2}{c|}{$I_m (mA)$} &
					\multicolumn{2}{c|}{$U_m (V)$}\\
				\hline

				$R_1$ & \ROne & $I_1$ & \IExTwoCalcOne & $U_1$ & \UExTwoCalcOne & $I_1$ & \IExTwoOne & $U_1$ & \UExTwoOne\\
				\hline

				$R_2$ & \RTwo & $I_2$ & \IExTwoCalcTwo & $U_2$ & \UExTwoCalcTwo & $I_2$ & \IExTwoTwo & $U_2$ & \UExTwoTwo\\
				\hline

				$R_3$ & \RMax & $I_3$ & \IExTwoCalcThree & $U_3$ & \UExTwoCalcTwo & $I_3$ & \IExTwoThree & $U_3$ & \UExTwoThree\\
				\hline
			\end{tabular} \end{center}
		
		\subsection{}
			\FPeval{\IExTwoKirchhoffDiff}{clip(\IExTwoTwo + \IExTwoThree - \IExTwoOne)}
			\FPround{\IExTwoKirchhoffDiff}{\IExTwoKirchhoffDiff}{2}

			\begin{minipage}{\textwidth}
				\begin{center} \textbf{Kirchhoff's Law} \end{center}
				\[
					\sum_{1}^{n} I_k = 0 \Rightarrow I_2 + I_3 - I_1 = 0 \Rightarrow \IExTwoTwo + \IExTwoThree - \IExTwoOne = \IExTwoKirchhoffDiff \approx 0
				\]
				\hrule
				\begin{center} \textbf{Power Balance} \end{center}

				%Left part
				\FPeval{\ExTwoPowerEqL}{clip(15 * \IExTwoOne)}
				\FPround{\ExTwoPowerEqL}{\ExTwoPowerEqL}{2}

				%Right part
				\FPeval{\ExTwoPowerEqR}{clip(\IExTwoOne * \IExTwoOne * (0 + \ROne))}
				\FPeval{\ExTwoPowerEqR}{clip(\ExTwoPowerEqR + \IExTwoTwo * \IExTwoTwo * \RTwo)}
				\FPeval{\ExTwoPowerEqR}{clip(\ExTwoPowerEqR + \IExTwoThree * \IExTwoThree * \RMax)}
				\FPeval{\ExTwoPowerEqR}{clip(\ExTwoPowerEqR / 1000)}
				\FPround{\ExTwoPowerEqR}{\ExTwoPowerEqR}{2}

				\[
					E \cdot I_1 = I_1 ^ 2 (r_0 + R_1) + I_2 ^ 2 R_2 + I_3 ^ 2 R_3 \Rightarrow
				\]
				\[
					15 \cdot \IExTwoOne = \frac{1}{10 ^ 3} \cdot (\IExTwoOne ^ 2 \cdot (0 + \ROne) + \IExTwoTwo ^ 2 \cdot \RTwo + \IExTwoThree ^ 2 \cdot \RMax)  \Leftrightarrow
				\]
				\[
					\ExTwoPowerEqL \ W \approx \ExTwoPowerEqR \ W
				\]
			\end{minipage}

		\subsection{}
			\begin{center} \begin{tabular}{|c|c|c|c|c|c|c|c|c|c|}
				\hline
				\multicolumn{7}{|c}{\textbf{Measured}} & \multicolumn{3}{|c|}{\textbf{Calculated}}\\
				\hline
				$R_3$ & U & $U_1$ & $U_2$ & $I_1$ & $I_2$ & $I_3$ & $U_1 + U_2$ & $I_2 + I_3$ & P\\
				\hline
				$\Omega$ & \multicolumn{3}{c|}{V} & \multicolumn{3}{c|}{mA} & V & mA & mW\\	
				\hline
				0 & \multirow{6}{*}{15} & 14.93 & 0.06 & 144.3 & 0.2 & 143.5 & 14.99 & 143.7 & 2164.5\\
				100 & & 9.06 & 5.88 & 88.4 & 29.4 & 59.2 & 14.94 & 88.6 & 1326\\
				200 & & 7.57 & 7.42 & 74 & 37.1 & 36.7 & 14.99 & 73.8 & 1110\\
				300 & & 6.9 & 8.0 & 67.2 & 40.5 & 26.7 & 14.9 & 67.2 & 1008\\
				467 & & 6.35 & 8.64 & 62 & 43.3 & 18.6 & 14.99 & 61.9 & 930\\
				\hline
			\end{tabular} \end{center}
			\[
				P_{R_3} = U \cdot I_{R_3} 
			\]

			\myIncludeImg{0.34}{./imgs/U1U2_I1I2I3_R3_plot.jpg}{}{graph1}

	\newpage
	\section{De cercetat proprietatile elementelor pasive (R, L, C) în circuitul de curent alternativ.}
		\myIncludeImg{0.7}{./imgs/ElectricCircuit3.jpg}{Circuit 2.1}{circuit3}
		\myIncludeImg{0.7}{./imgs/ElectricCircuit4.jpg}{Circuit 2.2}{circuit4}

		\begin{center} \begin{tabular}{|c|c|c|c|c|c|c|c|c|c|c|c|c|c|}
			\hline
			
			\multirow{2}{*}{Element} & U & $U_m$ & I & $I_m$ & $\phi$ & Q & S & P & R & C & L & $X_C$ & $X_L$\\
			\arrayrulecolor{gray!70}
			\cline{2-14}
			\arrayrulecolor{black}
			
			& \multicolumn{2}{c|}{V} & \multicolumn{2}{c|}{mA} & o & VAR & VA & W & $\Omega$ & nF & mH & $\Omega$ & $\Omega$\\
			\hline
			
			$R = 510 \Omega$ & & 10.3 & & 20 & & & & & & & & &\\
			\hline
			
			C = 56 nF & & 4.43 & & 186.3 & & & & & & & & &\\
			\hline
			
			L = 3.64 mH & & 11.34 & & 510 & & & & & & & & &\\
			\hline
		\end{tabular} \end{center}

		\myIncludeImg{0.1}{./imgs/Part2OscilographPos1.jpg}{Oscilograph pos 1}{oscil. 1}
		\myIncludeImg{0.1}{./imgs/Part2OscilographPos2.jpg}{Oscilograph pos 2}{oscil. 2}
		\myIncludeImg{0.1}{./imgs/Part2OscilographPos3.jpg}{Oscilograph pos 3}{oscil. 3}
\end{document}
