\documentclass{article}

\usepackage{enumitem}
\usepackage{listings}
\usepackage{color}
\usepackage{amsmath}
\usepackage{hyperref}
\usepackage{graphicx}
\usepackage{pgffor}
\usepackage{xparse}
\usepackage{expl3}
\usepackage{tabularx, makecell}
\usepackage{booktabs}
\usepackage{indentfirst}
\usepackage{lipsum}
\usepackage{sectsty}
\usepackage[utf8]{inputenc}
\usepackage{csquotes}
\usepackage{xcolor}
\usepackage{fancyvrb}

\graphicspath{{./}}

\definecolor{codegreen}{rgb}{0,0.6,0}
\definecolor{codegray}{rgb}{0.5,0.5,0.5}
\definecolor{codepurple}{rgb}{0.58,0,0.82}
\definecolor{backcolour}{rgb}{0.95,0.95,0.92}

\sectionfont{\bfseries\Large\center} 

\lstset{
	language=C++,
	basicstyle=\fontsize{5}{5}\ttfamily,
	keywordstyle=\color{blue},
	stringstyle=\color{orange},
	commentstyle=\color{gray},
	extendedchars=true,
	frame=single,
	keepspaces=true,
	numbers=left,
	numbersep=5pt,
	numberstyle=\color{gray},
	tabsize=4,
	morecomment=[l][\color{gray}]{\#}
}

\begin{document}
	% Custom Commands
	% Begin of the document

	\title{OOP laboratory\_02}
	\author{Terman Emil FAF161}
	\maketitle

	% Write at bottom
	\vspace*{\fill}
	
	\begin{flushright}
		PHD prof: M. Kulev
	\end{flushright}

	\begin{center}
		\LaTeX
	\end{center}

	\pagebreak

	\section{Subject}
		Constructor - initialization function
	\section{Objectives}
		\begin{itemize}
			\item Studierea principiilor de definire şi utilizare a constructorilor

			\item Studierea principiilor de def­­­inire şi utilizare a destructorilor
			
			\item Studierea tipurilor de constructori
		\end{itemize}

	\section{Task}
		\textbf{a)} Să se creeze clasa Date – dată cu cîmpurile: zi(1-28..31), lună(1-12), an (numere întregi). Să se definească constructorii; funcţiile membru de setare a zilei, lunii şi anului; funcţiile membru de returnare a zilei, lunii, anului; funcţiile de afişare: afişare tip \enquote{6 iunie 2004} şi afişare tip \enquote{6.06.2004}. Funcţiile de setare a cîmpurilor clasei trebuie să verifice corectitudinea parametrilor primiţi.

		\textbf{b)} Să se creeze clasa Matrix – matrice. Clasa conţine pointer spre int, numărul de rînduri şi de coloane şi o variabilă – codul erorii. Să se definească constructorul fără parametri (constructorul implicit), constructorul cu un parametru – matrice pătrată şi constructorul cu doi parametri – matrice dreptunghiulară ş. a. Să se definească funcţiile membru de acces: returnarea şi setarea valorii elementului (i, j). Să se definească funcţiile de adunare şi scădere a două matrice; înmulţirea unei matrice cu alta; înmulţirea unei matrice cu un număr. Să se testeze funcţionarea clasei. În caz de insuficienţă de memorie, necorespondenţă a dimensiunilor matricelor, depăşire a limitei memoriei utilizate să se stabilească codul erorii. 

	\section{Main notions of theory and used methods}
		A class in C++ is a user defined type or data structure declared with keyword class that has data and functions (also called methods) as its members whose access is governed by the three access specifiers private, protected or public (by default access to members of a class is private).
	\pagebreak

	\section{Data analysis}
		\subsection{Ex a}
			\lstinputlisting[language=C++]{./../ex00/includes/Date.hpp}
			\pagebreak

		\subsection{Ex b}
			\lstinputlisting[language=C++]{./../ex01/includes/Matrix.hpp}
			\pagebreak

\end{document}
