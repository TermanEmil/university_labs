\documentclass{article}

\usepackage{enumitem}
\usepackage{listings}
\usepackage{color}
\usepackage{amsmath}
\usepackage{hyperref}
\usepackage{graphicx}
\usepackage{pgffor}
\usepackage{xparse}
\usepackage{expl3}
\usepackage{tabularx, makecell}
\usepackage{booktabs}
\usepackage{indentfirst}
\usepackage{lipsum}
\usepackage{sectsty}
\usepackage[utf8]{inputenc}
\usepackage{csquotes}
\usepackage{xcolor}
\usepackage{fancyvrb}
\usepackage{fancyhdr}
\usepackage{fancyvrb}
\usepackage[most]{tcolorbox}
\usepackage{blindtext}

\graphicspath{{./}}

\definecolor{codegreen}{rgb}{0,0.6,0}
\definecolor{codegray}{rgb}{0.5,0.5,0.5}
\definecolor{codepurple}{rgb}{0.58,0,0.82}
\definecolor{backcolour}{rgb}{0.95,0.95,0.92}

\sectionfont{\bfseries\Large\center} 

\lstset{
	language=C++,
	frame=single,
	rulecolor=\color{gray},
	basicstyle=\fontsize{5}{5}\ttfamily,
	keywordstyle=\color{blue},
	stringstyle=\color{orange},
	commentstyle=\color{gray},
	extendedchars=true,
	keepspaces=true,
	numbers=left,
	numbersep=5pt,
	numberstyle=\color{gray},
	tabsize=4,
	morecomment=[l][\color{gray}]{\#}
}

\begin{document}
	% Custom Commands
	\newcommand{\cppInLine}[1]{
		{\lstinline[basicstyle=\small]|#1|}
	}
	% Begin of the document

	\title{OOP laboratory\_03}
	\author{Terman Emil FAF161}
	\maketitle

	% Write at bottom
	\vspace*{\fill}
	
	\begin{flushright}
		Conf. unv. prof: M. Kulev
	\end{flushright}

	\begin{center}
		\LaTeX
	\end{center}

	\pagebreak

	\section{Subject}
		Supraîncărcarea operatorilor
	\section{Objectives}
		\begin{itemize}
			\item Studierea necesităţii supraîncărcării operatorilor
			\item Studierea sintaxei de definire a operatorilor
			\item Studierea tipurilor de operatori
			\item Studierea formelor de supraîncărcare
		\end{itemize}

	\section{Task}
		\textbf{a)} Să se creeze o clasă de numere întregi. Să se definească operatorii "++" şi "+", ca metode ale clasei, iar operatorii "- -" şi "-" ca funcţii prietene. Operatorii trebuie să permită efectuarea operaţiilor atît cu variabilele clasei date, cît şi cu variabilele întregi de tip predefinit.

		\textbf{b)} Să se creeze o clasă Set – mulţimea numerelor întregi, utilizînd memoria dinamică. Să se definească operatorii de lucru cu mulţimile: "+" – uniunea, "*" – intersecţia, "-" scăderea, ca funcţii prietene, iar "+=" – înserarea unui nou element în mulţime, "==" – comparare la egalitate, ş. a. ca metode ale clasei. Să se definească operatorii "$<<$" şi "$>>$". Să se definească funcţia de verificare a apartenenţei unui element la o mulţime.

	\section{Main notions of theory and used methods}
		In programming, operator overloading, sometimes termed operator ad hoc polymorphism, is a specific case of polymorphism, where different operators have different implementations depending on their arguments. Operator overloading is generally defined by a programming language, a programmer, or both.
	\pagebreak

	\section{Data analysis}
		\subsection{Ex a}
			\lstinputlisting{./../ex00/includes/MyInt.hpp}

			\begin{itemize}
				\item
					\begin{enumerate}
						\item \cppInLine{MyInt operator + (MyInt const & target) const;}

						\item MyInt operator + (int nb) const;
					\end{enumerate}

					\par Constant overload operators, defined as class methods. The first is used on another instance of the same class, the second is used with a simple int.

				\item
					\begin{enumerate}
						\item \cppInLine{friend MyInt operator - (MyInt const & target1, MyInt const & target2);}
						\item \cppInLine{friend MyInt operator - (MyInt const & target, int nb);}
						\item \cppInLine{friend MyInt operator - (int nb, MyInt const & target);}
					\end{enumerate}
					\par Friend overload operators, used for each case.

				\item
					\begin{enumerate}
						\item \cppInLine{MyInt & operator ++ (void);}
						\item \cppInLine{MyInt operator -- (int);}
					\end{enumerate}
					\par The first is a method's overload operator, the second is defined as a friend operator.
			\end{itemize}
			\pagebreak

		\subsection{Ex b}
			\lstinputlisting[linerange = {1-88}]{./../ex01/includes/Set.tpp}
			\begin{itemize}
				\item
					\cppInLine{Set<T> &		operator = (Set<T> const & target);}

					\par Assign overload operator, which takes in a constant reference to the same class.

				\item
					\cppInLine{T &				operator [] (size_t const i) const;}

					\par Overload operator of squared braces, which returns a modifiable reference to the set's element.

				\item
					\begin{enumerate}
						\item
							\cppInLine{friend Set<U>	operator + (Set<U> const & target1, Set<U> const & target2);}
							\par \textbf{Union}
						\item
							\cppInLine{friend Set<U>	operator * (Set<U> const & target1, Set<U> const & target2);}
							\par \textbf{intersection}
						\item
							\cppInLine{friend Set<U>	operator - (Set<U> const & target1, Set<U> const & target2);}
							\par \textbf{Difference}
					\end{enumerate}

					\par Friend arithmetic overload operators, which take in 2 constant references of the same type of set.
			\end{itemize}

			\pagebreak

	\section{The actual code}
		\subsection{Ex a}
			\lstinputlisting{./../ex00/src/MyInt.cpp}

			\begin{minipage}{\textwidth}
				\begin{center}
					\textbf{main.cpp}
				\end{center}

				\lstinputlisting{./../ex00/src/main.cpp}
			\end{minipage}

			\begin{minipage}{\textwidth}
				\begin{center}
					\textbf{Output}
				\end{center}

				\lstset{language={}}
				\begin{lstlisting}
a + b = 8
a + 5 = 8
3 + b = 8
a - b = -2
a - 5 = -2
3 - b = -2
----------
++a: 4
a = 4
----------
a--: 4
a = 3
----------
int aInt = a; aInt = 3
				\end{lstlisting}
			\end{minipage}
			\pagebreak

		\subsection{Ex b}
			\lstinputlisting[firstline = 89, lastline = 361]{./../ex01/includes/Set.tpp}
			\clearpage

			\begin{samepage}
				\begin{center}
					\textbf{main.cpp}
				\end{center}

				\lstinputlisting{./../ex01/src/main.cpp}
			\end{samepage}

			\centering

			\begin{samepage}
				\begin{center}
					\textbf{Output}
				\end{center}

				\lstset{language={}}
				\lstinputlisting{./ex01_output}
			\end{samepage}
			\pagebreak

	\section{Analysis of the results and conclusions}
		\begin{itemize}
			\item \textit{Operators} are very useful in making the code more readable and more intuitive.

			\item Friend overload operators are useful when we want to make an action over a different class and the custom class. Ex: adding a float and a custom class.

			\item Friend overload operators are also useful when a++ or ++a is used.

			\item Using overload operators, we create some standarts: if a \textit{Matrix} class is created, then we would expect that this class supports addition, multiplication, etc.
		\end{itemize}

\end{document}
