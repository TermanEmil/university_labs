\documentclass{article}

\usepackage{caption}
\usepackage{verbatim}
\usepackage{xparse}

\usepackage{silence}
\WarningFilter{latex}{You have requested package}

\usepackage{lib/defaultReportSettings}
\usepackage{lib/myTitlePage}
\usepackage{lib/customPseudoLstling}
\usepackage{lib/customHyperRef}
\usepackage{lib/myFigure}
\usepackage{lib/myIncludeImg}

\newcounter{ResultCounter}
\DeclareCaptionLabelFormat{RSCaption}{\textbf{Result \arabic{ResultCounter}}}

% #1 captions
\NewDocumentEnvironment{myResult}{m}
{
	\begin{myFigure}
		\begin{flushleft}
}
{
		\end{flushleft}
		\refstepcounter{ResultCounter}
		\captionsetup{labelformat=RSCaption}
		\caption{#1}
	\end{myFigure}
}

\begin{document}
	\myTitlePage{MMC}{Terman Emil FAF161}[V. Turcan][Interpolations][4]

	\section{Quadratic and Cubic}
		\myIncludeImg{./imgs/Ex1_2.png}[0.7][Quadratic and Cubic plots]

		\par For these problems, \textbf{General Divided Difference} method was used.
		\par Given \(n + 1\) distinct points $x_0, ..., x_n$ with n $\geqslant$ 2, define:
		\[
			f[x_0, ..., x_n] =
				\frac{f[x_1, ..., x_n] - f[x_0, ..., x_{n-1}]}{x_n - x_0}
		\]
		\par This is a recursive definition of the $n^{th}$-order divided difference of $f(x)$, using divided differences of order n.

	\newpage
	\section{Splines}
		\myIncludeImg{./imgs/Ex3.png}[0.7][Splines]

		\par In this exercise, the same function, as in the previous exercise, was used. From this exercise, it can be seen how convinient are the interpolations: instead of simple lines, we recive a smooth function, which can be used in multiple problems, like finding the area or other stuff.
		\par From the graph, it can be seen the drawback of this method: in the area x = (0.2; 0.3) we have 2 points that are pretty close to each other (y = (33.5, 33)), resulting in a big deviation.

	\section{Matrices Gauss-Seidel}
		\begin{myResult}{Gauss-Seidel last iteration and actual solution}
			Last iteration:  [  8.03469022e+59  -1.60693804e+60]\\
			Solution:	 [ 1.  1.]\\
		\end{myResult}

		\par From the results, it can be seen that Gauss-Seidel method failed to find the solution, therefore the Gauss-Seidel method diverges when the roots are equal to 0.

	\section{Matrices equation solutions}
		\begin{myResult}{The solutions of the given equations}
			First system: No solution\\
			Second system: [(0, 0, 0, 1)]\\
		\end{myResult}
\end{document}