\documentclass{article}

\usepackage{caption}
\usepackage{verbatim}
\usepackage{xparse}

\usepackage{silence}
\WarningFilter{latex}{You have requested package}

\usepackage{lib/defaultReportSettings}
\usepackage{lib/myTitlePage}
\usepackage{lib/customPseudoLstling}
\usepackage{lib/customHyperRef}
\usepackage{lib/myFigure}

\newcounter{ResultCounter}
\DeclareCaptionLabelFormat{RSCaption}{\textbf{Result \arabic{ResultCounter}}}

% #1 captions
\NewDocumentEnvironment{myResult}{m}
{
	\begin{myFigure}
		\begin{flushleft}
}
{
		\end{flushleft}
		\refstepcounter{ResultCounter}
		\captionsetup{labelformat=RSCaption}
		\caption{#1}
	\end{myFigure}
}

\begin{document}
	\myTitlePage{MMC}{Terman Emil FAF161}[V. Turcan][][3]

	\section{Fixed point}
		\begin{myResult}{Fixed point iteration and Aitken method}
			\verbatiminput{./aux/ex01.results}
		\end{myResult}

		\par It can be clearly seen that Aitken extrapolation converges much faster (by the number of iterations). This algorithm is used to accelerate the rate of convergence of a sequence.

		\def \algOnePath {./aux/fixed_point_iteration.alg}
		\includeAlgFile{\algOnePath}[fixed\_point\_finder()]

		\includeAlgFile{./aux/aitken.alg}[aitken\_extrapolation()]

		\begin{center}
			\myHref{./../code/ex01.py}[Fixed point python code]
		\end{center}

	\section{Banking}
		\begin{myResult}{Banking results}
			\verbatiminput{./aux/ex02.results}
		\end{myResult}

		\par For this exercise, \textit{Newton} method was used.
		\par Knowing the at least the basics of numerical analysis, we can solve diverse problems, that seem to be a bit complicated at first.

		\begin{center}
			\myHref{./../code/ex02.py}[Banking python code]
		\end{center}
\end{document}