\documentclass{article}

\usepackage{caption}
\usepackage{karnaugh-map}
\usepackage{booktabs}

\usepackage{silence}
\WarningFilter{latex}{You have requested package}

\usepackage{lib/myIncludeImg}
\usepackage{lib/defaultReportSettings}
\usepackage{lib/myTitlePage}
\usepackage{lib/myFigure}

\newcounter{tableCounter}
\DeclareCaptionLabelFormat{TabCaption}{\textbf{Tab \arabic{tableCounter}}}

\def \tableOne {
	\begin{tabular}{|c|c|c|c|c||c|c|c|c|}
		\hline
		& $x_1$ & $x_2$ & $x_3$ & $x_4$ & $f_1$ & $f_2$ & $f_3$ & $f_4$\\
		\hline
		           & 8 & 6 & 1 & -4 & 4 & 3 & 2 & 1\\ \toprule[2pt]
		\textbf{0} & 0 & 0 & 0 &  0 & 0 & 0 & 0 & 0\\ \hline
		\textbf{1} & 0 & 0 & 1 &  0 & 0 & 0 & 0 & 1\\ \hline
		\textbf{2} & 0 & 1 & 0 &  1 & 0 & 0 & 1 & 0\\ \hline
		\textbf{3} & 0 & 1 & 1 &  1 & 0 & 0 & 1 & 1\\ \hline
		\textbf{4} & 1 & 0 & 0 &  1 & 0 & 1 & 0 & 1\\ \hline
		\textbf{5} & 1 & 0 & 1 &  1 & 1 & 0 & 0 & 1\\ \hline
		\textbf{6} & 0 & 1 & 0 &  0 & 1 & 0 & 1 & 0\\ \hline
		\textbf{7} & 0 & 1 & 1 &  0 & 1 & 1 & 0 & 0\\ \hline
		\textbf{8} & 1 & 0 & 0 &  0 & 1 & 1 & 0 & 1\\ \hline
		\textbf{9} & 1 & 0 & 1 &  0 & 1 & 1 & 1 & 1\\ \hline
	\end{tabular}
}

\def \tableTwo {
	\begin{tabular}{|c|c|}
		\hline
		f1 & f2\\ \hline
		\myKarnaugh{1,2,9,10,14}{
			\implicant{15}{10}
			\implicantedge{1}{3}{9}{11}
			\implicantedge{3}{2}{11}{10}
		}
		&
		\myKarnaugh{2,6,9,10}{
			\implicant{9}{11}
			\implicant{3}{6}
			\implicantedge{3}{2}{11}{10}
		}\\
		\toprule[3pt]
		f3 & f4\\ \hline
		\myKarnaugh{1,5,13,10}{
			\implicant{1}{7}
			\implicant{5}{15}
			\implicant{11}{10}
		}
		&
		\myKarnaugh{2,6,14,10,13,8}{
			\implicant{3}{10}
			\implicant{12}{14}
			\implicantedge{12}{8}{14}{10}
		}\\
		\hline
	\end{tabular}
}

% #1 minterms
% #2 connections
\DeclareDocumentCommand{\myKarnaugh}{m m}
{
	\begin{karnaugh-map}[4][4][1][$x_1x_2$][\rotatebox{90}{$x_3x_4$}]
		\minterms{#1}
		\terms{3,4,7,11,12,15}{$*$}
		\autoterms[0]

		#2
	\end{karnaugh-map}
}

\begin{document}
	\myTitlePage{ASDN}{Terman Emil FAF161}[S. Munteanu][Code convertors][2]

	{\Large \textbf{Objective:} Practical study of methods to convert codes}

	\def \imgOneCaption {Convertor Circuit: [8, 6, 1, -4] $\rightarrow$ [4, 3, 2, 1]}

	\myIncludeImg{./imgs/Circuit1.jpg}[0.5][\imgOneCaption]
	\myIncludeImg{./imgs/Circuit1Time.jpg}[0.8][Circuit timer]
	\begin{center}
		\(C = 34Q\)\\
		\(T_d = 2r\)\\
	\end{center}

	\begin{myFigure}
		\tableOne

		\refstepcounter{tableCounter}
		\captionsetup{labelformat=TabCaption}
		\caption{Convertion table}
	\end{myFigure}

	\begin{myFigure}
		\tableTwo

		\refstepcounter{tableCounter}
		\captionsetup{labelformat=TabCaption}
		\caption{Karnaugh maps for \textit{f1, f2, f3, f4}}
	\end{myFigure}

	\[
		a = x_1 \overline{x_4}
	\]
	\begin{tabular}{cc}
		\begin{minipage}{0.5\textwidth} \[
			f_1 = x_2 \overline{x_4} + x_1 \overline{x_4} + x_1 x_3
			= \overline{
				\overline{x_2 \overline{x_4}}
				\cdot
				\overline{a}
				\cdot
				\overline{x_1 x_3}
			}
		\] \end{minipage}
		&
		\begin{minipage}{0.5\textwidth} \[
			f_2 = x_1 \overline{x_4} + x_1 \overline{x_3} + x_2 x_3 \overline{x_4}
			= \overline{
				\overline{a}
				\cdot
				\overline{x_1 \overline{x_3}}
				\cdot
				\overline{x_2 x_3 \overline{x_4}}
			}
		\] \end{minipage}
		\\
		\begin{minipage}{0.5\textwidth} \[
			f_3 = x_2 \overline{x_3} + x_2 x_4 + x_1 x_3 \overline{x_4}
			= \overline{
				\overline{x_2 \overline{x_3}}
				\cdot
				\overline{x_2 x_4}
				\cdot
				\overline{x_1 x_3 \overline{x_4}}
			}
		\] \end{minipage}
		&
		\begin{minipage}{0.5\textwidth} \[
			f_4 = x_1 + x_3 x_4 + \overline{x_2} x_3
			= \overline{
				\overline{x_1}
				\cdot
				\overline{x_3 x_4}
				\cdot
				\overline{\overline{x_2} x_3}
			}
		\] \end{minipage}
	\end{tabular}

	\newpage
	{\Large \textbf{Conclusion:}}
		\par In this laboratory work I learned how to create binary convertors using only \textit{NAND} gates.
\end{document}