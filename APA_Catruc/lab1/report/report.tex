\documentclass{article}

\usepackage{enumitem}
\usepackage{listings}
\usepackage{color}
\usepackage{amsmath}
\usepackage{hyperref}
\usepackage{graphicx}
\usepackage{pgffor}
\usepackage{xparse}
\usepackage{expl3}
\usepackage{tabularx, makecell}
\usepackage{booktabs}
\usepackage{indentfirst}
\usepackage{lipsum}
\usepackage{sectsty}
\usepackage[utf8]{inputenc}
\usepackage{csquotes}
\usepackage{xcolor}
\usepackage{fancyvrb}
\usepackage{fancyhdr}
\usepackage{fancyvrb}
\usepackage[most]{tcolorbox}
\usepackage{blindtext}
\usepackage{caption}
\usepackage{etoolbox}

\graphicspath{{./}}

\definecolor{codegreen}{rgb}{0,0.6,0}
\definecolor{codegray}{rgb}{0.5,0.5,0.5}
\definecolor{codepurple}{rgb}{0.58,0,0.82}
\definecolor{backcolour}{rgb}{0.95,0.95,0.92}

\sectionfont{\bfseries\Large\center} 

% Lstlisting configuartions for C++
\lstset{
	language=C++,
	frame=single,
	rulecolor=\color{gray},
	basicstyle=\fontsize{5}{5}\ttfamily,
	keywordstyle=\color{blue},
	stringstyle=\color{orange},
	commentstyle=\color{gray},
	extendedchars=true,
	keepspaces=true,
	numbers=left,
	numbersep=5pt,
	numberstyle=\color{gray},
	tabsize=4,
	morecomment=[l][\color{gray}]{\#}
}

% Lstlisting configuartions for algorithm code
\newcounter{nalg}[section]
\renewcommand{\thenalg}{\arabic{nalg}}
\DeclareCaptionLabelFormat{algocaption}{\textbf{Algorithm \thenalg}}
\lstnewenvironment{algorithm}[1][]
{
    \refstepcounter{nalg}
    \captionsetup{labelformat=algocaption,labelsep=colon}
    \lstset{
        mathescape=true,
        frame=tB,
        numbers=left, 
        numberstyle=\tiny,
        basicstyle=\scriptsize, 
        keywordstyle=\color{blue}\bfseries\em,
        keywords={,input, output, return, datatype, function, in, if, else, foreach, while, begin, end, then, }
        numbers=left,
        xleftmargin=.04\textwidth,
        #1
    }
}{}

\begin{document}
	% Custom Commands

	% <Square cases>
	\makeatletter
	\newenvironment{sqcases} {
		\matrix@check\sqcases\env@sqcases
	}{
		\endarray \right.
	}
	\def\env@sqcases {
		\let \@ifnextchar \new@ifnextchar
		\left \lbrack
		\def \arraystretch{1.2}
		\array{@{}l@{\quad}l@{}}
	}
	\makeatother
	% </Square cases>

	% Begin of the document

	\title{APA laboratory\_01}
	\author{Terman Emil FAF161}
	\maketitle

	% Write at bottom
	\vspace*{\fill}
	
	\centering
	\includegraphics{imgs/UTM_logo.png}

	\begin{flushright}
		Prof: M. Catruc
	\end{flushright}

	\LaTeX
	\pagebreak

	\centering
	{\Huge \textbf{Subject:} Algorithm analyzing}
	
	\raggedright
	{\large \textbf{Purpose:}}
	\begin{itemize}
		\item Analiza empirică a algoritmilor.
		\item Analiza teoretică a algoritmilor.
		\item Determinarea complexităţii temporale şi asimptotice a algoritmilor
	\end{itemize}

	\raggedright
	{\large \textbf{Conditions:}}
	\begin{enumerate}
		\item Efectuaţi analiza empirică a algoritmilor propuşi.
		\item Determinaţi relaţia ce determină complexitatea temporală pentru aceşti algoritmi.
		\item Determinaţi complexitatea asimptotică a algoritmilor.
		\item Faceţi o concluzie asupra lucrării efectuate.
	\end{enumerate}

	\newpage
	\section{Recursive method}

	\begin{algorithm}[caption={Recursive method}, label={Fibonaci}]
function fib1(n)
	if n < 2 then
		return n
	else
		return fib1(n - 1) + fib1(n - 2)
	\end{algorithm}
	\begin{center} T(n) - ? \end{center}
	\par For line 2 and 3: O(1)
	\newline For line 5: T(n - 1) + T(n - 2)
	\newline So:
	\newline T(n) = 2, n $<$ 2
	\newline T(n) = T(n - 1) + T(n - 2) + 3 $\approx$ T(n - 1) + T(n - 2), n $\geq$ 2
	\begin{center}
		$t_{n} - t_{n - 1} - t_{n - 2} = 0$
		\par $x^2 - x - 1 = 0$
		\[
			\begin{sqcases}
				x_1 = \frac{1 - \sqrt{5}}{2}\\
				x_2 = \frac{1 + \sqrt{5}}{2}\\
			\end{sqcases}
		\]

		$t_n = C_1 (\frac{1 - \sqrt{5}}{2})^n + C_2 (\frac{1 + \sqrt{5}}{2})^n$
	\end{center}
	The fraction: \textit{$\frac{1 + \sqrt{5}}{2}$} is also known as the \textit{Golden Ratio} denoted as $\varphi$.

	The most significant part of $t_n$ is $\varphi$.

	\begin{center}
		$T(n) = O(\varphi^n)$
	\end{center}

	\newpage
	\section{Iterative method}

	\newpage
	\section{Optimized iterative method}

\end{document}
