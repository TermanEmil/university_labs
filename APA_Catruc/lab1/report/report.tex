\documentclass{article}

\usepackage{enumitem}
\usepackage{listings}
\usepackage{color}
\usepackage{amsmath}
\usepackage{hyperref}
\usepackage{graphicx}
\usepackage{pgffor}
\usepackage{xparse}
\usepackage{expl3}
\usepackage{tabularx, makecell}
\usepackage{booktabs}
\usepackage{indentfirst}
\usepackage{lipsum}
\usepackage{sectsty}
\usepackage[utf8]{inputenc}
\usepackage{csquotes}
\usepackage{xcolor}
\usepackage{fancyvrb}
\usepackage{fancyhdr}
\usepackage{fancyvrb}
\usepackage[most]{tcolorbox}
\usepackage{blindtext}
\usepackage{caption}
\usepackage{etoolbox}
\usepackage{geometry}
\usepackage{mdframed}
\usepackage{mathptmx}
\usepackage{anyfontsize}
\usepackage{t1enc}

\graphicspath{{./}}

\definecolor{codegreen}{rgb}{0,0.6,0}
\definecolor{codegray}{rgb}{0.5,0.5,0.5}
\definecolor{codepurple}{rgb}{0.58,0,0.82}
\definecolor{backcolour}{rgb}{0.95,0.95,0.92}

\sectionfont{\bfseries\Large\center} 

% Lstlisting configuartions for C++
\lstset{
	language=C++,
	frame=single,
	rulecolor=\color{gray},
	basicstyle=\fontsize{5}{5}\ttfamily,
	keywordstyle=\color{blue},
	stringstyle=\color{orange},
	commentstyle=\color{gray},
	extendedchars=true,
	keepspaces=true,
	numbers=left,
	numbersep=5pt,
	numberstyle=\color{gray},
	tabsize=4,
	morecomment=[l][\color{gray}]{\#}
}

% Lstlisting configuartions for algorithm code
\newcounter{nalg}
\renewcommand{\thenalg}{\arabic{nalg}}
\DeclareCaptionLabelFormat{algocaption}{\textbf{Algorithm \thenalg}}
\lstnewenvironment{algorithm}[1][]
{
    \refstepcounter{nalg}
    \captionsetup{labelformat=algocaption,labelsep=colon}
    \lstset{
        mathescape=true,
        frame=tB,
        numbers=left, 
        numberstyle=\tiny,
        basicstyle=\scriptsize, 
        keywordstyle=\color{blue}\bfseries\em,
        keywords={,input,output,return,datatype,function,in,if,else,foreach,while,begin,end,then,to,do,for,div,\mod,mod,array,int,of,declare,}
        numbers=left,
        xleftmargin=.04\textwidth,
        #1
    }
}{}

\geometry{
	a4paper,
	total={170mm,257mm},
	left=20mm,
	top=20mm,
}

\begin{document}
	%Custom Commands

	% My include img. (With frame and caption) 
	\newcommand{\myIncludeImg}[4]{
		\begin{center} \begin{figure}[!ht]
			\begin{mdframed}[backgroundcolor=black!5, rightline=false, leftline=false]
				\begin{center}
					\includegraphics[scale=#1]{#2}
					\caption{#3}
				\end{center}
			\end{mdframed}
			\label{fig:#4}
		\end{figure} \end{center}
	}

	\newcommand{\myCenter}[1]{
		\begin{center}
			#1
		\end{center}
	}

	% <Square cases>
	\makeatletter
	\newenvironment{sqcases} {
		\matrix@check\sqcases\env@sqcases
	}{
		\endarray \right.
	}
	\def\env@sqcases {
		\let \@ifnextchar \new@ifnextchar
		\left \lbrack
		\def \arraystretch{1.2}
		\array{@{}l@{\quad}l@{}}
	}
	\makeatother
	% </Square cases>

	% Begin of the document
	\begin{center}
		Technical University of Moldova\\
		Inignerical department S.A.

		\vfill {\fontsize{100}{1}\selectfont Report}\\
		{\huge Analiza si proiectarea algoritmilor}\\
		{\huge \textbf{Subject:} Algorithm analyzing - Fibbonaci}
		\vfill
	\end{center}
	Author: \hfill Terman Emil FAF161\\
	Prof: \hfill M. Catruc
	\vfill
	\begin{center}
		Chisinau 2017
	\end{center}
	\pagebreak

	\begin{center}
		{\Huge \textbf{Subject:} Algorithm analyzing}
	\end{center}

	{\large \textbf{Purpose:}}
	\begin{itemize}
		\item[--] analiza empirică a algoritmilor.
		\item[--] analiza teoretică a algoritmilor.
		\item[--] determinarea complexităţii temporale şi asimptotice a algoritmilor
	\end{itemize}

	{\large \textbf{Conditions:}}
	\begin{enumerate}
		\item[--] efectuaţi analiza empirică a algoritmilor propuşi.
		\item[--] determinaţi relaţia ce determină complexitatea temporală pentru aceşti algoritmi.
		\item[--] determinaţi complexitatea asimptotică a algoritmilor.
		\item[--] faceţi o concluzie asupra lucrării efectuate.
	\end{enumerate}

	\newpage
	\section{Recursive method}

	\begin{algorithm}[caption={Recursive method}, label={Fibonacci1}]
function fib1(n)
	if n < 2 then
		return n
	else
		return fib1(n - 1) + fib1(n - 2)
	\end{algorithm}
	\begin{center} T(n) - ? \end{center}
	\par For line 2 and 3: O(1)
	\par For line 5: T(n - 1) + T(n - 2)
	\par So:
	\par T(n) = 2, n $<$ 2
	\par T(n) = T(n - 1) + T(n - 2) + 3 $\approx$ T(n - 1) + T(n - 2), n $\geq$ 2
	\begin{center}
		$t_{n} - t_{n - 1} - t_{n - 2} = 0$
		\par $x^2 - x - 1 = 0$
		\[
			\begin{sqcases}
				x_1 = \frac{1 - \sqrt{5}}{2}\\
				x_2 = \frac{1 + \sqrt{5}}{2}\\
			\end{sqcases}
		\]

		$t_n = C_1 (\frac{1 - \sqrt{5}}{2})^n + C_2 (\frac{1 + \sqrt{5}}{2})^n$
	\end{center}
	\par The fraction: \textit{$\frac{1 + \sqrt{5}}{2}$} is also known as the \textit{Golden Ratio} denoted as $\varphi$.
	The most significant part of $t_n$ is $\varphi$, thus:

	\myCenter{$T(n) = O(\varphi^n)$}

	\par This method is very uneficient since it has to recalculate multiple timess the same values. We can clearly see why it's not a good idea to use this algorithm:

	\myIncludeImg{0.5}{./imgs/PlotFib1.png}{Algorithm 1}{graph1}

	\newpage
	\section{Iterative method}
		\begin{algorithm}[caption={Iterative method}, label={Fibonacci2}]
function fib2(n)
	$i \leftarrow 1;$
	$j \leftarrow 0;$
	for $k \leftarrow 1$ to n do
		$j \leftarrow i + j;$
		$i \leftarrow j - i;$
	return $j$
		\end{algorithm}

		\par For line 2 and 3, the time is O(1).
		\par Line 5 and 6 are executed n times, therefore the time for both of these lines will be 2 $\cdot$ n.
		\par We consider line 4 to be executed n times.
		\newline
		\[
			t_n = 2 + n + 2n
		\]
		\par Therefore:
		\[
			T(n) = O(n)
		\]

		\myIncludeImg{0.5}{./imgs/PlotFib2.png}{Algorithm 2}{graph2}

	\newpage
	\section{Logarithmic method}
		\begin{algorithm}[caption={Logarithmic method}, label={Fibonacci3}]
function fib3(n)
	$i \leftarrow 1;$
	$j \leftarrow 0;$
	$k \leftarrow 0;$
	$h \leftarrow 1;$
	while $n > 0$ do
		if $n$ mod $2 == 1$ then
			$t \leftarrow j \cdot h;$
			$j \leftarrow i \cdot h + j \cdot k + t;$
			$i \leftarrow i \cdot k + t;$
		$t \leftarrow h \cdot h;$
		$h \leftarrow 2 \cdot k \cdot h + t;$
		$k \leftarrow k \cdot k + t;$
		$n \leftarrow n$ div $2;$
	return $j$
		\end{algorithm}

		\par From the line 14, we can see that the \textcolor{blue}{while} loop will be executed $\log_2 n$ times, due to the operation:
		\[
			n \leftarrow n \ div \ 2
		\]
		\par Therefore:
		\[
			T(n) = log_2 n
		\]

		\myIncludeImg{0.5}{./imgs/PlotFib3.png}{Algorithm 3}{graph3}

		\begin{center}
			Those steps are generated when the \textit{if} statement is executed.
		\end{center}

	\newpage
	\section{Iterative with saved values}
		\begin{algorithm}[caption = {Iterative with saved values}, label = {Fibonacci4}]
function fib4(n)
	declare fibVals : array[1, n + 2] of int;

	$fibVals[0] \leftarrow 0;$
	$fibVals[1] \leftarrow 1;$
	for $i \leftarrow 2$ to $n + 1$ do
		$fibVals[i] = fibVals[i - 1] + fibVals[i - 2];$
	return fibVals[n];
		\end{algorithm}

		\begin{itemize}
			\item[--] for the line 2, we declare (n + 2) $\cdot$ \textit{\textcolor{red}{sizeof}(\textcolor{blue}{int})} = (n + 2) * 4 \textit{Bytes} of memory. That's roughly 4 $\cdot$ n \textit{Bytes}.
			\item[--] line 4 and 5 have a complexity of O(n).
			\item[--] the line 7 has 4 operations, that is another O(n).
			\item[--] the \textcolor{blue}{\textit{for}} loop is executed $n - 1$ times. So the inside part of the loop will have $4 \cdot (n - 1)$ operations. Resulting that the entire loop will have complexity of O(n).
		\end{itemize}
		\par Since the most significant part is the \textcolor{blue}{\textit{for}} loop, results that:
		\[
			T(n) = O(n)
		\]
		\myIncludeImg{0.5}{./imgs/PlotFib4.png}{Algorithm 4}{graph4}

		\begin{center}
			\par This method has a very big drawback: it needs more memory than other algorithms.
		\end{center}

	\newpage
	\section{Summary}
		\myIncludeImg{0.4}{./imgs/PlotFibAll.png}{All algorithms}{graph5}
		\myIncludeImg{0.4}{./imgs/PlotFibAllButFact.png}{All algorithms but factorial}{graph6}

		In These graphs we can see how big is the difference between each algorithm.

	\newpage
	\section{Conclusion}
		At this laboratory work, I compared different algorithms and found the best one, that is \(O(\log_2 n)\). This comparison helped me visualize how different are the algorithms, which made me draw the folowing conclusions:
		\begin{itemize}
			\item[--] the fastest and the most eficient Fibonacci algorithm is the $O(\log_2 n)$ algorithm.
			\item[--] sometimes, it's enough to have a very simple implemented algorithm, but less efficient, like the recursive method, because we don't always need a huge performance.
			\item[--] the calculation of the time complexity of an algorithm helps us to choose which is the best algorithm of all.
			\item[--] an algorithm is also described by how much memory it needs. For example, the last method consumes more memory as n grows, making this algorithm less attractive.
		\end{itemize}
\end{document}
